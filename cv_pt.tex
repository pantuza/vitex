%%%%%%%%%%%%%%%%%%%%%%%%%%%%%%%%%%%%%%%%%
%
% This template was modified by Gustavo Pantuza as in
% http://github.com/pantuza/vitex
%
% Plasmati Graduate CV
% LaTeX Template
% Version 1.0 (24/3/13)
%
% This template has been downloaded from:
% http://www.LaTeXTemplates.com
%
% Original author:
% Alessandro Plasmati (alessandro.plasmati@gmail.com)
%
% License:
% CC BY-NC-SA 3.0 (http://creativecommons.org/licenses/by-nc-sa/3.0/)
%
% Important note:
% This template needs to be compiled with XeLaTeX.
% The main document font is called Fontin and can be downloaded for free
% from here: http://www.exljbris.com/fontin.html
%
%%%%%%%%%%%%%%%%%%%%%%%%%%%%%%%%%%%%%%%%%




%----------------------------------------------------------------------------------------
%	PACKAGES AND OTHER DOCUMENT CONFIGURATIONS
%----------------------------------------------------------------------------------------

\documentclass[a4paper,10pt]{article} % Default font size and paper size

\usepackage{fontspec} % For loading fonts
\defaultfontfeatures{Mapping=tex-text}
\setmainfont[SmallCapsFont = Fontin SmallCaps]{Fontin} % Main document font

\usepackage{xunicode,xltxtra,url,parskip} % Formatting packages

\usepackage[usenames,dvipsnames]{xcolor} % Required for specifying custom colors

\usepackage[big]{layaureo} % Margin formatting of the A4 page, an alternative to layaureo can be \usepackage{fullpage}
% To reduce the height of the top margin uncomment: \addtolength{\voffset}{-1.3cm}

\usepackage{hyperref} % Required for adding links	and customizing them
\definecolor{linkcolour}{rgb}{0,0.2,0.6} % Link color
\hypersetup{colorlinks,breaklinks,urlcolor=linkcolour,linkcolor=linkcolour} % Set link colors throughout the document

\usepackage{titlesec} % Used to customize the \section command

\usepackage{longtable} % Used for footnotes inside tables

\titleformat{\section}{\Large\scshape\raggedright}{}{0em}{}[\titlerule] % Text formatting of sections
\titlespacing{\section}{0pt}{3pt}{3pt} % Spacing around sections

\begin{document}

\pagestyle{empty} % Removes page numbering

\font\fb=''[cmr10]'' % Change the font of the \LaTeX command under the skills section




%----------------------------------------------------------------------------------------
%	NAME AND CONTACT INFORMATION
%----------------------------------------------------------------------------------------

\par{\centering{\Huge Gustavo \textsc{Pantuza}}\bigskip\par} % Your name

\section{Dados Pessoais}

\begin{tabular}{rl}
\textsc{Data de Nascimento:} & 12 de Outubro de 1989 \\
\textsc{Localidade:} & Rio de Janeiro, RJ, Brasil \\
\textsc{Telefone:} & +55 31 9 8367 2657\\
\textsc{email:} & \href{mailto:gustavopantuza@gmail.com}{gustavopantuza@gmail.com}
\end{tabular}




%----------------------------------------------------------------------------------------
%	WORK EXPERIENCE
%----------------------------------------------------------------------------------------

\section{Experiências Profissionais}

\begin{tabular}{r|p{11cm}}

\textsc{Atual} & \emph{Engenheiro de software sênior na Globo.com}, \\
& \footnotesize{Trabalhou no núcleo de publicação de conteúdo que integra
todos os serviços internos resultando na entrega rápida de páginas.
Trabalhou também na criação de roterador de camada 4 via \emph{software} para
habilitar serviços genéricos na nuvem. Trabalhou no projeto
\emph{Globo Network API} que mantém o estado e as configurações de rede do
\emph{Datacenter}.} \\
\multicolumn{2}{c}{} \\

\textsc{Fev 2012 - Dec 2014} & \emph{ Mestre em Ciência da Computação pela
    Universidade Federal de Minas Gerais (UFMG). Pesquisador em
Redes e Sistemas Distribuídos}, \\
& \footnotesize{Desenvolvimento e avaliação de módulos em controladores SDN
e sistemas distribuídos em C, C++ e Python. Propôs uma visão topológica e
global de Redes Definidas por Software baseada em um modelo de grafos.} \\
\multicolumn{2}{c}{} \\

\textsc{Set 2012 - Jun 2013} & \emph{Engenheiro de Software e Administrador
de Sistemas na Latitude 14} \\
& \footnotesize{Gerenciou um time desenvolvimento na construção de
aplicações Web utilizando as metodologias Scrum e TDD.
Construiu a arquitetura e desenho do sistema.
Responsável pelo \emph{deploy}, ambiente de desenvolvimento e todos
os serviços cuja aplicação possuía dependência. } \\
\multicolumn{2}{c}{} \\

%------------------------------------------------

\textsc{Ago 2010-Set 2012} & Engenheiro de Software na \textsc{Studio Sol
Comunicação Digital}  \\
& \footnotesize{Construiu aplicações Web, de tempo real e
APIs para aplicações móveis utilizando Python/Django, MySQL,
Javascript, html/css e PHP.
Administrou um banco de dados distribuído e sistemas como
Lucene Solr, Nginx, MongoDB, Redis, uWSGI}\\
\multicolumn{2}{c}{} \\

%------------------------------------------------

\textsc{Jan 2009 - Ago 2010} & Analista de Teste de Software
na \textsc{Totvs S.A} \\
& \footnotesize{Definiu processos de Teste de software utilizando uma
junção das metodoligas Interativo/Incremental e Espiral visando
gerência de configuração, automatização de testes e documentação.}
\end{tabular}




%----------------------------------------------------------------------------------------
%	EDUCATION
%----------------------------------------------------------------------------------------

\section{Formação}

\begin{tabular}{rl}
\textsc{Dezembro} 2014 & Mestrado em Ciência da Computação na área de
\textsc{Sistemas Distribuídos}
\\ & na (UFMG)
\textbf{Universidade Federal de Minas Gerais} \\
& Dissertação: ``Aspectos, Arquitetura e Implementação de controladores
\\ & Distribuídos em Redes Definidas por Softwar''
| \small Orientador: Prof. Luiz \textsc{F M Vieira} \\
&\\

%------------------------------------------------

\textsc{Dezembro} 2012& Bacharel em \textsc{Ciência da Computação}
pelo \normalsize\textbf{Centro Universitário de}
\\ & \textbf{Belo Horizonte}, UNIBH \\
& Projeto de Conclusão: ``Monitoramento e cuidado ao Idoso baseado em
\\ & Sistemas Distribuídos e Microcontroladores''
| \small Orientador: Euzébio \textsc{Souza} \\
&\\

%------------------------------------------------

\textsc{Dezembro} 2007 & Técnico em Informática Industrial,
\textbf{Escola Técnica Juscelino Kubitschek}, Ipatinga \\
&\\

%------------------------------------------------

\textsc{Julho} 2007 & Língua Inglesa, \textbf{Uptime Consultants}
\end{tabular}




%----------------------------------------------------------------------------------------
%	SCHOLARSHIPS AND ADDITIONAL INFO
%----------------------------------------------------------------------------------------

\section{Comunidades e Constribuições}

\begin{tabular}{rl}
\textsc{OpenSUSE}  & Membro da Comunidade OpenSUSE \\
\textsc{Python Brasil}  & Membro da Associação Python Brasil \\
\textsc{Mininet}  & Contribuiu adicionando ambientes de simulação de redes
dinâmicas \\
\textsc{FsF} & Membro da Free Software Foundation \\
\textsc{Área31 Hackerspace} & Membro fundador do
\href{http://area31.net.br}{Hackerspace} \\
\textsc{Pox} & Adicionou uma representação em grafos da rede com a
\\ & finalidade de gerenciamento em redes \\
\textsc{Projetos Pessoais} & \href{http://github.com/pantuza}{Projetos}
pessoais em Software Livre \\
\end{tabular}




%----------------------------------------------------------------------------------------
%	LANGUAGES
%----------------------------------------------------------------------------------------

%\section{Languages}

%\begin{tabular}{rl}
%\textsc{English:} & Fluent\\

%\textsc{Italian:} & Mothertongue\\

%\textsc{French:} & Basic Knowledge\\
%\end{tabular}




%----------------------------------------------------------------------------------------
%	COMPUTER SKILLS
%----------------------------------------------------------------------------------------

\section{Habilidades em Computação}

\begin{longtable}{rl}
Linguagens: & Python, C, C++, bash, javascript, PHP,
Java, html/css, {\fb \LaTeX}
\footnote{Esse documento foi gerado utilizando {\fb \LaTeX} e está
disponível como um \href{http://github.com/pantuza/vitex}{projeto}
de Software Livre no Github.} \\
Bancos de Dados: & MySQL, PostgreSQL, MongoDB, Redis \\
Sistemas Operacionais: & OpenSUSE, CentOS, Ubuntu, Fedora, Arch Linux \\
Serviços: & nginx, Apache, Tornado, memcached, Lucene Solr, git,
\\ & squid, bind9, uWSGI, Cron \\
Ferramentas de Rede: & nmap, tcpdump, iptraf, iperf \\
Ferramentas de Desenvolvimento: & vim, gdb, valgrind, ipdb, nose, pylint, POSIX tools \\
\end{longtable}




%----------------------------------------------------------------------------------------
%	INTERESTS AND ACTIVITIES
%----------------------------------------------------------------------------------------

\section{Atividades e Interesses}
Visite meu site \href{https://pantuza.com}{https://pantuza.com}
e conheça meu código, assim como meu convívio em redes sociais e meus
interesses.
Tenho o hábito de escrever artigos sobre ciência da computação em meu
blog \href{https://blog.pantuza.com}{https://blog.pantuza.com}

%----------------------------------------------------------------------------------------
\end{document}
