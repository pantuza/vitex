%%%%%%%%%%%%%%%%%%%%%%%%%%%%%%%%%%%%%%%%%
%
% This template was modified by Gustavo Pantuza as in
% http://github.com/pantuza/vitex
%
% Plasmati Graduate CV
% LaTeX Template
% Version 1.0 (24/3/13)
%
% This template has been downloaded from:
% http://www.LaTeXTemplates.com
%
% Original author:
% Alessandro Plasmati (alessandro.plasmati@gmail.com)
%
% License:
% CC BY-NC-SA 3.0 (http://creativecommons.org/licenses/by-nc-sa/3.0/)
%
% Important note:
% This template needs to be compiled with XeLaTeX.
% The main document font is called Fontin and can be downloaded for free
% from here: http://www.exljbris.com/fontin.html
%
%%%%%%%%%%%%%%%%%%%%%%%%%%%%%%%%%%%%%%%%%




%----------------------------------------------------------------------------------------
%	PACKAGES AND OTHER DOCUMENT CONFIGURATIONS
%----------------------------------------------------------------------------------------

\documentclass[a4paper,10pt]{article} % Default font size and paper size

\usepackage{fontspec} % For loading fonts
\defaultfontfeatures{Mapping=tex-text}
\setmainfont[SmallCapsFont = Fontin SmallCaps]{Fontin} % Main document font

\usepackage{xunicode,xltxtra,url,parskip} % Formatting packages

\usepackage[usenames,dvipsnames]{xcolor} % Required for specifying custom colors

\usepackage[big]{layaureo} % Margin formatting of the A4 page, an alternative to layaureo can be \usepackage{fullpage}
% To reduce the height of the top margin uncomment: \addtolength{\voffset}{-1.3cm}

\usepackage{hyperref} % Required for adding links	and customizing them
\definecolor{linkcolour}{rgb}{0,0.2,0.6} % Link color
\hypersetup{colorlinks,breaklinks,urlcolor=linkcolour,linkcolor=linkcolour} % Set link colors throughout the document

\usepackage{titlesec} % Used to customize the \section command

\usepackage{longtable} % Used for footnotes inside tables

\titleformat{\section}{\Large\scshape\raggedright}{}{0em}{}[\titlerule] % Text formatting of sections
\titlespacing{\section}{0pt}{3pt}{3pt} % Spacing around sections

\begin{document}

\pagestyle{empty} % Removes page numbering

\font\fb=''[cmr10]'' % Change the font of the \LaTeX command under the skills section




%----------------------------------------------------------------------------------------
%	NAME AND CONTACT INFORMATION
%----------------------------------------------------------------------------------------

\par{\centering{\Huge Gustavo \textsc{Pantuza}}\bigskip\par} % Your name

\section{Dados Pessoais}

\begin{tabular}{rl}
\textsc{Data de Nascimento:} & 12 de Outubro de 1989 \\
\textsc{Localidade:} & Rio de Janeiro, RJ, Brasil \\
\textsc{Telefone:} & +55 31 9 8367 2657\\
\textsc{email:} & \href{mailto:gustavopantuza@gmail.com}{gustavopantuza@gmail.com}
\end{tabular}




%----------------------------------------------------------------------------------------
%	WORK EXPERIENCE
%----------------------------------------------------------------------------------------

\section{Experiências Profissionais}

\begin{tabular}{r|p{11cm}}

\textsc{Atual} & \emph{\bf Engenheiro de software sênior na Globo.com}, \\
& \footnotesize{Trabalho em um projeto de software livre escrito em Python
    chamado \href{https://github.com/globocom/GloboNetworkAPI}{Network API}.
    É um serviço de \emph{cloud} que tem visão global sobre a infraestrutura
    de redes. Tem controle sobre cada IP alocado, recurso utilizado e
    provisionamento de equipamentos. Na Globo.com é utilizado para controlar
    todo o \emph{datacenter}. No ambiente interno de \emph{cloud} integra-se
    com CloudStack, Xen, Tsuru, DBaaS entre outros projetos privados em nuvem.

    Anteriormente, trabalhei no Globo Core, um time responsável pela aplicação
    que faz a entrega final de páginas Web (CDA - \emph{Content delivery
    application}) e também no ciclo de vida do conteúdo (CMA - \emph{Content
    management application}). O primeiro deve ser o mais rápido possível e
    é utilizado nos principais portais da Globo.com como G1, Gshow e Globo
    Esporte. O segundo é uma plataforma Web onde jornalistas e editores
    escrevem o conteúdo.} \\
\multicolumn{2}{c}{} \\

\textsc{Fev 2012 - Dec 2014} & \emph{\bf Mestre em Ciência da Computação pela
    Universidade Federal de Minas Gerais (UFMG). Pesquisador em
Redes e Sistemas Distribuídos}, \\
& \footnotesize{Como pesquisador pelo Conselho Nacional de Desenvolvimento
    Científico e Tecnológico (CNPq) eu escrevi a seguinte dissertação: Grafos
    como uma primitiva do plano de controle para análise e gerenciamento de
    Redes Definidas por Software.

    Durante esse período cursei matérias como Projeto e análise de algoritmos,
    processamento de dados massivos, Algoritmos distribuídos, Engenharia de
    aplicações em rede, Sistemas Operacionais, Redes inteligentes e Padrões de
    programação paralela.

    Dentro do laboratório WiNET, escrevi um programa balanceador de carga IP
    através do controlador SDN POX. Além disso, criei um módulo em grafos que
    mantém uma visão global da topologia de redes e escuta eventos de entrada
    e saída de máquinas e enlaces na rede. O grafo serve como uma ferramenta de
    gerenciamento.} \\
\multicolumn{2}{c}{} \\

\textsc{Set 2012 - Jun 2013} & \emph{\bf Engenheiro de Software e Administrador
de Sistemas na Latitude 14} \\
& \footnotesize{Trabalhei com uma equipe dedicada de 7 pessoas em um portal de
    música independente. Nesse projeto privado para a Telefonica Brasil S/A
    (Vivo) participei da implantação de servidores através do AWS EC2, Cloud
    Front, RDS, S3, Route 53 com aplicações em Django e Postgres. Como time,
    criamos, mantivemos e evoluímos o projeto. Participei de muitas reuniões
    com investidores para discutir futuro e definir metas incrementais.} \\
\multicolumn{2}{c}{} \\

%------------------------------------------------

\textsc{Ago 2010-Set 2012} & \emph{\bf Engenheiro de Software na
Studio Sol Comunicação Digital}  \\
& \footnotesize{Criei APIs e aplicações Web com Python/Django, MySQL,
    Javascript, html/css e PHP para 3 produtos: Letras, Cifra-club e PalcoMP3.
    O Letras tem 30 milhões pageviews/dia. Cifra-club 10 milhões/dia. O
    aplicativo do PalcoMP3 tem 100 milhões de \emph{downloads} no Google play.
    Administrei também projetos como Lucene Solr, Nginx, MongoDB, Redis, uWSGI
    em infraestrutura de servidores dedicados
}\\
\multicolumn{2}{c}{} \\

%------------------------------------------------

\textsc{Jan 2009 - Ago 2010} & \emph{\bf Analista de Teste de Software
na Totvs S.A} \\
& \footnotesize{Participei do desenho de processos de testes dentro de
metodologias interativo/incremental com foco em gerência de configuração,
automação de \emph{build} e documentação.}
\end{tabular}


\footnotetext{Mais experiências descritas no
\href{https://www.linkedin.com/in/gustavo-pantuza-46089825/}{perfil do Linkedin}} 


%----------------------------------------------------------------------------------------
%	EDUCATION
%----------------------------------------------------------------------------------------

\section{Formação}

\begin{tabular}{rl}
\textsc{Dezembro} 2014 & Mestrado em Ciência da Computação na área de
\textsc{Redes e Sistemas Distribuídos}
\\ & na (UFMG)
\textbf{Universidade Federal de Minas Gerais} \\
& Dissertação: ``Grafos como uma primitiva do plano de controle para análise e
\\ & gerenciamento de redes definidas por software''
| \small Orientador: Prof. Luiz \textsc{F M Vieira} \\
&\\

%------------------------------------------------

\textsc{Dezembro} 2012& Bacharel em \textsc{Ciência da Computação}
pelo \normalsize\textbf{Centro Universitário de}
\\ & \textbf{Belo Horizonte}, UNIBH \\
& Projeto de Conclusão: ``Monitoramento e cuidado ao Idoso baseado em
\\ & Sistemas Distribuídos e Microcontroladores''
| \small Orientador: Euzébio \textsc{Souza} \\
&\\

%------------------------------------------------

\textsc{Dezembro} 2007 & Técnico em Informática Industrial,
\textbf{Escola Técnica Juscelino Kubitschek}, Ipatinga \\
&\\

%------------------------------------------------

\textsc{Julho} 2007 & Língua Inglesa, \textbf{Uptime Consultants}
\end{tabular}




%----------------------------------------------------------------------------------------
%	SCHOLARSHIPS AND ADDITIONAL INFO
%----------------------------------------------------------------------------------------

\section{Comunidades e Constribuições}

\begin{tabular}{rl}
\textsc{OpenSUSE}  & Membro da Comunidade OpenSUSE \\
\textsc{Python Brasil}  & Membro da Associação Python Brasil \\
\textsc{Mininet}  & Contribuiu adicionando ambientes de simulação de redes
dinâmicas \\
\textsc{FsF} & Membro da Free Software Foundation \\
\textsc{Área31 Hackerspace} & Membro fundador do
\href{http://area31.net.br}{Hackerspace} \\
\textsc{Pox} & Adicionou uma representação em grafos da rede com a
\\ & finalidade de gerenciamento em redes \\
\textsc{Projetos Pessoais} & \href{http://github.com/pantuza}{Projetos}
pessoais em Software Livre \\
\end{tabular}




%----------------------------------------------------------------------------------------
%	LANGUAGES
%----------------------------------------------------------------------------------------

%\section{Languages}

%\begin{tabular}{rl}
%\textsc{English:} & Fluent\\

%\textsc{Italian:} & Mothertongue\\

%\textsc{French:} & Basic Knowledge\\
%\end{tabular}




%----------------------------------------------------------------------------------------
%	Base de conhecimento
%----------------------------------------------------------------------------------------

\section{Base de conhecimento}

\begin{longtable}{rl}
Linguagens: & Python, C, C++, bash, javascript, Java, PHP, html/css, {\fb \LaTeX}
\footnote{Esse documento foi gerado utilizando {\fb \LaTeX} e está
disponível como um \href{http://github.com/pantuza/vitex}{projeto}
de Software Livre no Github.} \\
Bancos de Dados: & MySQL, PostgreSQL, MongoDB, Redis \\
Sistemas Operacionais: & OpenSUSE, CentOS, Ubuntu, Fedora, Arch Linux \\
Serviços: & nginx, memcached, Lucene Solr, squid, bind9, uWSGI, Cron \\
Ferramentas de Rede: & nmap, tcpdump, iptraf, iperf \\
Ferramentas de Cloud: & KVM, Docker, LXC, Cloud Stack \\
Ferramentas de Desenvolvimento: & vim, git, gdb, valgrind, POSIX tools \\
Ciência da Computação: & Algoritmos, Redes, Sistemas distribuídos, Sistemas operacionais \\
\end{longtable}




%----------------------------------------------------------------------------------------
%	INTERESTS AND ACTIVITIES
%----------------------------------------------------------------------------------------

\section{Atividades e Interesses}
Visite meu site \href{https://pantuza.com}{https://pantuza.com}
e conheça meu código, assim como meu convívio em redes sociais e meus
interesses.
Tenho o hábito de escrever artigos sobre ciência da computação em meu
blog \href{https://blog.pantuza.com}{https://blog.pantuza.com}

%----------------------------------------------------------------------------------------
\end{document}
