%%%%%%%%%%%%%%%%%%%%%%%%%%%%%%%%%%%%%%%%%
%
% This template was modified by Gustavo Pantuza as in
% http://github.com/pantuza/vitex
%
% Plasmati Graduate CV
% LaTeX Template
% Version 1.0 (24/3/13)
%
% This template has been downloaded from:
% http://www.LaTeXTemplates.com
%
% Original author:
% Alessandro Plasmati (alessandro.plasmati@gmail.com)
%
% License:
% CC BY-NC-SA 3.0 (http://creativecommons.org/licenses/by-nc-sa/3.0/)
%
% Important note:
% This template needs to be compiled with XeLaTeX.
% The main document font is called Fontin and can be downloaded for free
% from here: http://www.exljbris.com/fontin.html
%
%%%%%%%%%%%%%%%%%%%%%%%%%%%%%%%%%%%%%%%%%




%----------------------------------------------------------------------------------------
%	PACKAGES AND OTHER DOCUMENT CONFIGURATIONS
%----------------------------------------------------------------------------------------

\documentclass[a4paper,10pt]{article} % Default font size and paper size

\usepackage{fontspec} % For loading fonts
\defaultfontfeatures{Mapping=tex-text}
\setmainfont[SmallCapsFont = Fontin SmallCaps]{Fontin} % Main document font

\usepackage{xunicode,xltxtra,url,parskip} % Formatting packages

\usepackage[usenames,dvipsnames]{xcolor} % Required for specifying custom colors

\usepackage[big]{layaureo} % Margin formatting of the A4 page, an alternative to layaureo can be \usepackage{fullpage}
% To reduce the height of the top margin uncomment: \addtolength{\voffset}{-1.3cm}

\usepackage{hyperref} % Required for adding links	and customizing them
\definecolor{linkcolour}{rgb}{0,0.2,0.6} % Link color
\hypersetup{colorlinks,breaklinks,urlcolor=linkcolour,linkcolor=linkcolour} % Set link colors throughout the document

\usepackage{titlesec} % Used to customize the \section command

\usepackage{longtable} % Used for footnotes inside tables
\usepackage{makecell} % Allow to break lines inside one cell

\titleformat{\section}{\Large\scshape\raggedright}{}{0em}{}[\titlerule] % Text formatting of sections
\titlespacing{\section}{0pt}{3pt}{3pt} % Spacing around sections

\begin{document}

\pagestyle{empty} % Removes page numbering

\font\fb=''[cmr10]'' % Change the font of the \LaTeX command under the skills section




%----------------------------------------------------------------------------------------
%	NAME AND CONTACT INFORMATION
%----------------------------------------------------------------------------------------

\par{\centering{\Huge Gustavo \textsc{Pantuza}}\bigskip\par} % Your name

\section{Personal Data}

\begin{tabular}{rl}
\textsc{Place and Date of Birth:} & Brazil | 12 October 1989 \\
\textsc{Address:} & Belo Horizonte - MG, Brazil \\
\textsc{Phone:} & +55 31 9 8367 2657\\
\textsc{email:} & \href{mailto:gustavopantuza@gmail.com}{gustavopantuza@gmail.com}
\end{tabular}




%----------------------------------------------------------------------------------------
%	WORK EXPERIENCE
%----------------------------------------------------------------------------------------

\section{Work Experience}

\begin{tabular}{r|p{11cm}}

\textsc{Current} & \emph{\bf Senior software engineer at Globo.com}, \\
& \footnotesize{I work at a Free Software project written in Python called
	\href{https://github.com/globocom/GloboNetworkAPI}{Network API}.
    It is a Cloud service that has a global view of a network infrastructure.
    It controls every IP network, resource allocation and network
    hardware provisioning. At Globo.com, we use it to control the entire
    data center. In our cloud environment, we integrate it with cloud stack,
    Xen, tsuru, DBaaS and some private cloud services.

    We developed a cloud solution inside Network API through OpenDaylight,
    OpenFlow and OpenVSwitch to automatize and deploy Access Control Lists on a
    Xen Cluster.} \\

& \footnotesize{
    Previously, I've worked at the Globo Core team that is responsible for the
    Content Delivery Application (CDA) and Content Management Application
    (CMA). The first one delivers contents directly to end users and must be
    as fastest as possible. It is a software in front of all major Globo.com
    products like G1, GShow and GloboEsporte. The second is a web based
    platform in which every editor, journalist or reporter use to write
    products contents.} \\
\multicolumn{2}{c}{} \\

%------------------------------------------------

\textsc{Feb 2012 - Dec 2014} & \emph{\bf Master's in Computer Science at
Federal University of Minas Gerais (UFMG). Researcher in
Networks and Distributed Systems.}, \\
& \footnotesize{As a researcher by the Nacional Center of Research in Brazil
    (CNPq) I've written the master thesis with the following title: Graphs as
    a primitive on the control plane to analyse and manage software defined
    networks.

    With in the course I've took classes of Project and analysis of algorithms,
    Distributed algorithms, Massive data processing, Network application
    engineering, Operation systems, Intelligent networks and Patterns of
    parallel programing.} \\

& \footnotesize{
    Working inside WiNET lab, I've written a IP Load Balancer through Pox SDN
    Controller. Also, add a Graph module to
    Pox Controller that can keep a global view of a network topoloy and monitor
    link discovery, host join and leave network. The graph serves as a network
    management tool.} \\
\multicolumn{2}{c}{} \\

%------------------------------------------------

\textsc{Sep 2012 - Jun 2013} & \emph{\bf Software engineer and System
Administrator at Latitude 14} \\
& \footnotesize{Worked with a dedicated team of 7 people on a Independent
	music festival web site. At this private project for Telefonica Brasil
	S/a (Vivo), I deployed servers with AWS EC2, Cloud front, RDS, S3,
	Route 53 and written in Django with Postgres. As a team, we design,
	implement and maintain the entire software. Also, we use to meet our
	stakeholders and discuss project evolution, define incremental releases
	with milestones.} \\
\multicolumn{2}{c}{} \\

%------------------------------------------------

\textsc{Aug 2010 - Sep 2012} & \emph{\bf Software Engineer at Studio Sol
Comunicação Digital}  \\
& \footnotesize{Built Mobile APIs and Web applications with Python/Django,
	MySQL, Javascript, html/css and PHP for three products: Letras, Cifra-club
	and Palco MP3. Letras has 30 million pageviews/day. Cifra-club 10 million
	pageviews/day and PalcoMP3 mobile app has 100 millions downloads on
	Android. Also, I've managed systems like Lucene Solr, Nginx, MongoDB,
	Redis, uWSGI.} \\
\multicolumn{2}{c}{} \\

%------------------------------------------------

\textsc{Jan 2009 - Aug 2010} & \emph{\bf Software Test Analyst/Engineer
at Totvs S.A} \\
& \footnotesize{Designed test processes and softwares using a mix
of iterative/incremental and spiral methodologies looking at
configuration management, build automation and documentation.} \\
\multicolumn{2}{c}{} \\

\end{tabular}

\footnotetext{More experiences described at
\href{https://www.linkedin.com/in/gustavo-pantuza-46089825/}{Linkedin Profile}} 

%----------------------------------------------------------------------------------------
%	EDUCATION
%----------------------------------------------------------------------------------------

\section{Education}

\begin{tabular}{rl}
\textsc{December} 2014 & Master's in Computer Science in
\textsc{Networks and Distributed Systems},  \textbf{Federal University of}
\\ & \textbf{Minas Gerais}, UFMG \\
% & 110/110 \small\emph{First Class Honours} | Major: Quantitative Finance\\
& Thesis: ``Graphs as primitive of the control plane for management and
\\ & analysis of software defined networks''
| \small Advisor: Prof. Luiz \textsc{F M Vieira} \\
% &\normalsize \textsc{Gpa}: 8.0/9.0\hyperlink{grds}{\hfill | \footnotesize Detailed List of Exams}\\
&\\

%------------------------------------------------

\textsc{December} 2012& Undergraduate Degree in
\textsc{}\textsc{Computer Science},
% \\&110/110 \small\emph{Commerce Specialization},
\normalsize\textbf{ University Center of Belo}
\\ & \textbf{Horizonte}, UNIBH \\
& Conclusion Project: ``Monitoring and Aged Care System Based on
Distributed
\\ & Systems and Microcontrollers''
| \small Advisor: Euzébio \textsc{Souza} \\
% &\normalsize \textsc{Gpa}: 7.5/9.0 \hyperlink{grds_usc}{\hfill| \footnotesize Detailed List of Exams}\\
&\\

%------------------------------------------------

\textsc{December} 2007 &Technical computation for industries,
\textbf{Juscelino Kubitschek Technical}
\\ & \textbf{School}, Ipatinga \\
% & \textsc{Gpa}: 8.0/9.0 \hyperlink{grds_usc}{\hfill| \footnotesize Detailed List of Exams}\\
&\\

%------------------------------------------------

\textsc{July} 2007 & English language, \textbf{Uptime Consultants}
\end{tabular}




%----------------------------------------------------------------------------------------
%	SCHOLARSHIPS AND ADDITIONAL INFO
%----------------------------------------------------------------------------------------

\section{Communities and Contributions}

\begin{tabular}{rl}
\textsc{OpenSUSE}  & Member of OpenSUSE community \\
\textsc{Python Brazil}  & Member of Python Brazil Association \\
\textsc{Mininet}  & Contributed adding Dynamic Network environment \\
\textsc{FsF} & Member of Free Software Foundation \\
\textsc{Área31 Hackerspace} & Founding Member of the
\href{http://area31.net.br}{Hackerspace} \\
\textsc{Pox} & Added a Graph representation of the network with
\\ & management purpose \\
\textsc{Personal Projects} & Personal Free Software
\href{http://github.com/pantuza}{projects} \\
\end{tabular}




%----------------------------------------------------------------------------------------
%	LANGUAGES
%----------------------------------------------------------------------------------------

%\section{Languages}

%\begin{tabular}{rl}
%\textsc{English:} & Fluent\\

%\textsc{Italian:} & Mothertongue\\

%\textsc{French:} & Basic Knowledge\\
%\end{tabular}




%----------------------------------------------------------------------------------------
%	Knowledge base
%----------------------------------------------------------------------------------------

\section{Knowledge base}

\begin{longtable}{rl}
Languages: & Python, C, C++, bash, javascript, Java, PHP, html/css, {\fb \LaTeX}
\footnote{This Document was generated using {\fb \LaTeX} and is
available as a Free Software
\href{http://github.com/pantuza/vitex}{project} hosted at Github.} \\
Databases: & MySQL, PostgreSQL, MongoDB, Redis \\
Operating System: & OpenSUSE, CentOS, Ubuntu, Fedora, Arch Linux \\
Services: & nginx, memcached, Lucene Solr, squid, bind9, uWSGI, Cron \\
Networking tools: & nmap, tcpdump, iptraf, iperf \\
Cloud tools: & KVM, Docker, LXC, Cloud Stack \\
Development tools: & vim, git, gdb, valgrind, POSIX tools \\
Computer Science: & Algorithms, Network, Distributed systems, Operating systems \\
\end{longtable}


%----------------------------------------------------------------------------------------
%	Publications
%----------------------------------------------------------------------------------------

\section{Publications}

\begin{longtable}{rl}
    \href{https://ieeexplore.ieee.org/document/7014202}{CNSN 2014}: & \makecell[l]{Network Management through Graphs in Software Defined Networks} \\
\href{http://www.sbrc2014.ufsc.br/anais/files/wpeif/anaisWPEIF2014.pdf}{SBRC 2014}: & \makecell[l]{[pt-BR] Análise e Gerenciamento de Rede através de Grafos \\
    em Redes Definidas por Software} \\
\href{https://www.dcc.ufmg.br/pos/cursos/defesas/1824M.PDF}{Master thesis}: & \makecell[l]{[pt-BR] Graphs as a primitive of the control plane to network analisys \\
    and management through Software Defined Networking} \\
\end{longtable}





%----------------------------------------------------------------------------------------
%	INTERESTS AND ACTIVITIES
%----------------------------------------------------------------------------------------

\section{Interests and Activities}
visit my site \href{https://pantuza.com}{https://pantuza.com} and check out my
social networks profiles and interests.
I am used to write computer science articles in brazilian portuguese in my
personal blog \href{https://blog.pantuza.com}{https://blog.pantuza.com}.

%----------------------------------------------------------------------------------------
\end{document}
